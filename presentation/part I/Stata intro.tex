\documentclass[pdftex, compress]{beamer}

\usepackage{%
	marvosym,%
	graphicx,%
	hyperref,%
	verbatim,%
}

\usepackage[T1]{fontenc}
\usepackage{fancyvrb}
\usepackage[english]{babel}
\usepackage[labelformat=empty, labelsep=none, justification=centering]{caption}

\usetheme{Boadilla}

\mode<presentation>
\setbeamercovered{transparent=30}
%%%%
\title{An introduction to Stata}
\author{Matthijs~de~Zwaan}
\institute{\,}
\date{November 19, 2010}
%%%%
\begin{document}
%%%%
\begin{frame}
\titlepage
\end{frame}
%%%%
\section{Introduction}

\begin{frame}
\frametitle{Introduction}
\begin{itemize}
	\item Stata has broad statistical and econometric capabilities
	\begin{itemize}
		\item Linear models; ANOVA
		\item Limited dependent variables (logit, tobit etc)
		\item Panel data and time series
		\item Survival analysis
		\item etc\dots
	\end{itemize}
	\item Good graphical capabilities (publication quality)
	\item Data Management
\end{itemize}
\end{frame}
%%%%
\section{Getting started}
\begin{frame}
\frametitle{Getting started}
%\insertnavigation
\begin{itemize}
	\item Point--and-Click via the menu
	\item Type directly into command--box
	\item (Better) alternative: Stata \texttt{.do} files
	\begin{itemize}
		\item Organise your do--files
		\item \alert{Reproduce} what you've done
		\item Easily checked and changed
		\item You can more easily split your work into blocks: one do for cleaning data, one for generating variables, one for analysis etc
	\end{itemize}
	\item[] $\Rightarrow$ \texttt{doedit} opens a new do--file	
\end{itemize}
\end{frame}

\begin{frame}
\frametitle{The \texttt{.do} file}
The do file\dots
\begin{itemize}
	\item \dots works as if you would type each line into the command box
	\item \dots can be run line by line, or all at once.
\end{itemize}
\vfill
\alert{Describe your do file!} 
\begin{itemize}
\item[] What does (should\dots) it do,
\item[] when was it made, 
\item[] by whom, 
\item[] why?
\end{itemize}
\end{frame}

\begin{frame}[fragile]
\frametitle{The structure of Stata commands}
A Stata command generally looks like:

\begin{Verbatim}
[(optional by, xi, etc)[, options]:] 
command (and what to do the command on) 
[restrictions: if, in] 
[, options]
\end{Verbatim}
Example:
\begin{verbatim}
by firm, sort: reg wage educ tenure if male==0, vce(robust)
\end{verbatim}

$\Rightarrow$ \alert{Most commands (and variables) can be abbreviated:}\\handy but dangerous! Use \texttt{set varabbrev off}.
\end{frame}

\begin{frame}
\frametitle{Notes on notation}
\begin{itemize}
	\item Lists of variables (\texttt{varlist}s)
	\begin{itemize}
		\item wildcard *: \texttt{var*} gives all variables starting with \texttt{var} (including \texttt{var})
		\item single character wildcard ?: \texttt{var?} gives all four character variables starting with var (\emph{excluding} \texttt{var}!)
		\item Can be used in the middle of a name
		\item var1-var6 gives all variables from var1 to var6. \alert{NB!} depends on sort order of your data set!
	\end{itemize}	
		\item Comments (generally within do-files)
	\begin{itemize}
		\item asterisk * at the beginning of a line
		\item // at the end of a line
		\item anything between /* comment */
	\end{itemize}
\end{itemize}
\end{frame}

\begin{frame}
	\frametitle{Need help?}
	\begin{itemize}
		\item Check the help file! \\ (\texttt{help [command]})
		\item Check the manuals!
		\item Check the internet!\\\url{http://www.ats.ucla.edu/stat/stata/} is a good resource. See also the Stata list.
	\end{itemize}
\end{frame}

\begin{frame}
\frametitle{Stata at work\dots}
\Huge Let's get to work!
\end{frame}

%%%%
\end{document}