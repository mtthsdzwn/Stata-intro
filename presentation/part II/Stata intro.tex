% plug ins: http://www.stata.com/plugins/index.html 
% database: http://www.stata.com/support/faqs/data/plugin_database.html


\documentclass[pdftex, compress]{beamer}

\usepackage{%
	marvosym,%
	graphicx,%
	hyperref,%
	verbatim,%
	fancyvrb,%
}

\usepackage[T1]{fontenc}
\usepackage{fancyvrb}
\usepackage[english]{babel}
\usepackage[labelformat=empty, labelsep=none, justification=centering]{caption}

\usetheme[secheader]{Boadilla}

\mode<presentation>
\setbeamercovered{transparent=30}
%%%%
\title{An introduction to Stata, part II}
\subtitle{Advanced use and programming}
\author{Matthijs~de~Zwaan}
\institute{ACED}
\date{December 3, 2010}
%%%%
\begin{document}
%%%%
\begin{frame}
\titlepage
\end{frame}
%%%%
\begin{frame}
\tableofcontents
\end{frame}
%%%%
\section{Loops}

\begin{frame}[fragile]
\frametitle{\texttt{foreach}, \texttt{forval} and \texttt{while}}
	\begin{itemize}
		\item \texttt{foreach\{\}}, \texttt{forval\{\}} and \texttt{while\{\}} allow you to repeat manipulations easily.
		\item useful for data manipulation or doing the same analyses over different subsets of your data.  
		\item For example, \\
			\begin{Verbatim}
				foreach i in 0 1	{
					summarize wage if gender == `i'
				}
			\end{Verbatim}
			summarizes the ``wage'' variable for men and women (presumably)
		\item take care of the syntax; see the help files!
	\end{itemize}
\end{frame}

\section{Using your results}

\begin{frame}
\frametitle{Macros, scalars and matrices}
	\begin{itemize}
		\item Macros can store a list of things (both text and number)
		\begin{itemize}
			\item can be used, for example, in a \texttt{foreach\{\}} loop
		\end{itemize}
		\item Scalars hold a single number
		\begin{itemize}
			\item can be used in further calculations; scalars are very precise!
		\end{itemize}
		\item Matrices hold arrays of scalars
		\begin{itemize}
			\item can be used in further calculations
		\end{itemize}		
	\end{itemize}
\end{frame}

\begin{frame}
\frametitle{Access to results}
	\begin{itemize}
		\item Stata stores results of commands in its memory
		\item you can access those results for further calculation/manipulation
		\item useful if you want to verify something, if or if the statistic you want is not directly shown and you need to calculate it from the available information.
		\item at the bottom of any help file, there is a list of things that are accessible
		\item see \texttt{help return}, and links therein
		\item Results are returned as matrices, scalars or macros
	\end{itemize}
\end{frame}

\section{``Pretty'' output and graphs}

\begin{frame}
\frametitle{Outputting your output}
	\begin{itemize}
		\item export your results to Excel/Word/\LaTeX
		\item Useful functions and SSC--packages:
		\begin{itemize}
			\item \texttt{estimate} functions --- see \texttt{help estimates}
			\item outreg2 --- SSC
			\item estout, esttab, estpost, eststo --- SSC
		\end{itemize}
	\end{itemize}
\end{frame}

\begin{frame}
\frametitle{Publication quality graphs}
	\begin{itemize}
		\item You can customize Stata's \texttt{graph twoway} commands in \alert{many} ways
		\item Multiple axes, legend, labeling of data points, colours, add text and notes, etc 
		\item see \texttt{help graph twoway} and \texttt{help twoway options}
		\item see Mitchell (2004), \textit{A visual guide to Stata graphics}.
	\end{itemize}
\end{frame}

\section{Make your own data}

\begin{frame}
\frametitle{Random number generation}
You can use Stata to generate (pseudo--)random number for a number of distributions
	\begin{itemize}
		\item You know the \alert{true} coefficients!
		\item You need to use \texttt{set obs \#} to define the number of observations your data will hold
		\item use \texttt{set seed} to get the same ``random'' number each time: for reproducability
		\item useful for testing, teaching
		\item see \texttt{help random number functions}
	\end{itemize}
\end{frame}

\section{Packages}

\begin{frame}
\frametitle{Don't invent the wheel again!}
	\begin{itemize}
		\item There are many user-written packages available
		\item Statistical Software Components (SSC) archive or \url{http://www.stata.com/stb/}: triple--checked by high quality Stata users, with good quality help files
		\item \alert{DO NOT install manually!}
		\item use the \texttt{findit} command to search for a package and install it
		\item or use \texttt{ssc install <\textsl{package name}>} to install
		\item see also \texttt{help ssc}
	\end{itemize}
\end{frame}

\section{Programs and \texttt{.ado} files}

\begin{frame}
\frametitle{Programs}
	\begin{itemize}
		\item If you use sequence of commands regularly, you can make a program, which you can then easily use again.
		\item If you save your program as an \texttt{.ado} file, you can then call the program any time, as you would any Stata command
		\item type \texttt{sysdir} to see where your \texttt{.ado}-files are saved. 
		\item or install \texttt{adoedit} from SSC.
		\item see \texttt{help program} for syntax. Have a look at any of the \texttt{.ado}-files in your directory for inspiration!
	\end{itemize}
\end{frame}

\section{Capita Selecta}

\begin{frame}
\frametitle{Capita Selecta}
\begin{itemize}
	\item Linking with data base programs: see \texttt{help odbc} and \url{http://www.stata.com/support/faqs/data/plugin_database.html}
\end{itemize}
\end{frame}
\end{document}
